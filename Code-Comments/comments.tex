\documentclass{article}
\usepackage{xspace}
%\usepackage{fancyvrb}

\begin{document}


\newcommand{\code}[1]{{\tt #1}\xspace}


\newcommand{\cKPABE}{\code{KPABE}}
\newcommand{\cSS}{\code{SecretSharing}}
\newcommand{\cAP}{\code{AccessPolicy}}
\newcommand{\cST}{\code{ShareTuple}}
\newcommand{\cBLcanon}{\code{BLSS}}
\newcommand{\cBLcanonAP}{\code{BLAccessPolicy}}
\newcommand{\cShTree}{\code{ShTreeSS}}
\newcommand{\cShTreeAP}{\code{ShTreeAccessPolicy}}
\newcommand{\cnode}{\code{NodeContent}}
\newcommand{\ctree}{\code{TreeNode}}
\newcommand{\cBig}{\code{Big}}
\newcommand{\cPFC}{\code{PFC}}

\newcommand{\cstring}{\code{std::string}}

\newcommand{\fmake}{\code{make}}
\newcommand{\fhkpabe}{\code{kpabe.h}}
\newcommand{\fckpabe}{\code{kpabe.cpp}}
\newcommand{\ftkpabe}{\code{testkpabe.cpp}}
\newcommand{\fhss}{\code{secretsharing.h}}
\newcommand{\fcss}{\code{secretsharing.cpp}}
\newcommand{\fhblcanon}{\code{BLcanonical.h}}
\newcommand{\fcblcanon}{\code{BLcanonical.cpp}}
\newcommand{\ftblcanon}{\code{testBLcanonical.cpp}}
\newcommand{\fhshtree}{\code{ShTree.h}}
\newcommand{\fcshtree}{\code{ShTree.cpp}}
\newcommand{\ftshtree}{\code{testShTree.cpp}}
\newcommand{\fhtree}{\code{tree.h}}
\newcommand{\fctree}{\code{tree.cpp}}
\newcommand{\fttree}{\code{testtree.cpp}}
\newcommand{\fhutils}{\code{utils.h}}
\newcommand{\fcutils}{\code{utils.cpp}}
\newcommand{\ftutils}{\code{testutils.cpp}}


%\title{Implementation of an ABE scheme from [Crampton,Pinto - CSF'14]}
%\maketitle



\section{Introduction}

This document describes an implementation of a generic Key Policy Attribute Based Encryption (KPABE) scheme, as detailed in the paper by Crampton and Pinto in CSF'14. The driving idea is that such a scheme can be implemented using an arbitrary linear secret sharing scheme (LSSS) as a building block. This implementation project had the following objectives:
\begin{itemize}
\item provide an implementation of a generic scheme based on any LSSS;
\item compare the performance of two basic implementation schemes: Benaloh-Leichter with canonical policies (canonical BL); a tree of Shamir schemes (Shamir Tree);
\end{itemize}

The language chosen to implement was C++. The main reason was the availability of the library MIRACL to perform cryptographic computations and operations on big numbers, by advice of a member of the project (Kenny Patterson), since it is freely available, open source, fast and is perhaps the best available implementation of pairing-based cryptography. This suggested two possible languages: plain C and C++. C++ was chosen because of its object-oriented nature, that better suits the main idea of the scheme: modularity of the secret sharing scheme within the ABE scheme.

As much as possible, the code was developed with a Test-Driven Development methodology (TDD), meaning that tests should be developed before the implementation of the actual code to  be tested. This was not always achieved, but at the worst the tests were developed in parallel with the code.
These tests run on the console and use colour coded output to quickly signal the success or failure of tests.

The implementation is divided in different types of files:
\begin{itemize}
\item header files with the declaration of functions and classes;
\item implementation files of said classes. These are compiled into object files only, not into executables;
\item testfiles, that link with the relevant object files and produce executables to run the tests;
\item a benchmark file, that produces executables to obtain performance data;
\item a makefile to produce all the object and executable code;
\item some configuration files that hold declarations similar to those in header files, but are separated from these to provide variant; compilations. This is necessary to produce distinct benchmark files from the same source code.
\end{itemize}



\section{Architecture}

The central notions of the data model are the concepts of KPABE and of LSSS scheme, and are implemented in their respective classes: \cKPABE and \cSS. The class \cSS requires other supporting classes. A secret sharing scheme implements one access policy of a specific type. The notion of access policy is captured by the class \cAP. The function of a secret sharing scheme is to produce shares of a given secret, and to reconstruct a secret from a set of shares. Shares are held by participants of the scheme, in a many-to-one relationship. The reconstruction algorithm can be performed by anyone with access to the shares. If the access policy is not trivial, this operation usually requires some information about the share, besides its share value proper, that is known by all participants: the share's public information. Thus, the concept of share is not representable by a single primitive type. Therefore, it is captured by a class called \cST.

Class \fckpabe is declared in file \fhkpabe and defined in file \fckpabe. The file \ftkpabe holds tests for this construction. The class \cKPABE itself is not abstract, but it can only be instantiated with an instance of a concrete secret sharing scheme. The test file runs a battery of tests for two instantiations of \cKPABE, one for each of the two secret sharing schemes currently implemented in this project.

The classes \cST, \cAP and \cSS are all declared in file \fhss and defined in file \fcss. Due to the generality of the concepts they represent, the classes \cSS and \cAP are abstract, since some of their properties can only be implemented in specific concrete instances.

This implementation provides, so far, two concrete implementations: a Benaloh-Leichter scheme for canonical policies; and a tree of simple Shamir threshold schemes. From this base it is easy to provide a simple Shamir threshold scheme (there is an old implementation of that in the \code{DeadCode} directory, that no longer compiles, in files \code{shamir2.h}, \code{shamir2.cpp} and \code{testshamir2.cpp}) and a generalized Benaloh-Leichter, that accepts a tree of \code{AND}, \code{OR} and possibly \code{THRESHOLD} gates. There is also preliminary work in the DeadCode section, in files \code{BL.h}, \code{BL.cpp} and \code{testBL.cpp}.

The canonical BL scheme is implemented by the classes \cBLcanon and \cBLcanonAP, both declared in file \fhblcanon and defined in file \fcblcanon.
Accompanying tests are implemented in file \ftblcanon. Analogously, the tree of Shamir schemes is captured by classes \cShTree and \cShTreeAP, implemented and tested in the corresponding files \fhshtree, \fcshtree, \ftshtree.

Because of its tree-like nature, a Shamir Tree policy is best represented by a tree data structure. There is a natural difference between leaf and inner nodes of a tree, but in this case the information that needs to be associated to the leaf nodes is specifically different of that in the inner nodes. For that reason, I developed a tree data structure specific for this project. It is captured by two classes: \cnode describes a particular node, and its associated data, without regards to its position in a tree; \ctree describes a tree as a collection of nodes related in a specific way. Both these classes are declared in \fhtree and defined in \fctree. The respective tests are in file \fttree.

Some operations are repeatedly needed throughout the code that were not readily available in the language. These have been implemented aside as ``utility'' functions, in files \fhutils and \fcutils, with some tests in file \ftutils. These include the declaration of all header files needed by the project, definition of codes to write in colour for a terminal, definitions of macros for writing messages to the terminal in a quick fashion, mostly used for debugging, a flag that tells whether to ignore debug output messages, some error codes for exception messages, constants defining how to represent policy operators, functions to report the result of tests, and some more specialized functions. These will be described below in more detail.


\section{Implementation Details}

\subsection{Using MIRACL}
\label{sec_miracl}
This project implements a cryptographic scheme for use in the real-world. Since its security is based on number-theoretic complexity assumptions, it requires very large numbers that can not be handled by the primitive types and functions of C++. Therefore, implementation relies on an external library, MIRACL, to provide these services. The current version is 4.0.

In particular, it uses MIRACL's ability to manipulate large integers, and operations on asymmetric pairings of groups $G_1 \times G_2 \to G_T$.

MIRACL provides different instantiations of pairing curves, with varying levels of AES-comparable security from 80 bits to 256 bits. For efficiency reasons, we chose Barreto-Naehrig curves for a 128-bit security level, which is currently considered secure in practice. 

To specify this choice, the code must define two macros:
\begin{verbatim}
#define MR_PAIRING_BN    // Barreto-Naehrig curves
#define AES_SECURITY 128 // AES-128 security
\end{verbatim}

This is defined at the start of \fhutils, which is then ultimately included by all source files in the project.
In order to use the pairing related functions, this file also includes the header \verb|pairing_3.h|, which gives the interface for Type 3 asymmetric group pairings.

Code with MIRACL in C++ requires special care. Any variable declared in C++ is immediately initialized at the point of declaration with some default value. In the case of classes, like \cBig, this means invoking a default constructor. Therefore, MIRACL must be initialized at the start of any source code that will generate an executable file. In this project, that means some of the test files and the benchmarks.

MIRACL is essentially a C library, that also provides a C++ interface. The way to initialize in C is to invoke \code{mirsys} to obtain a Miracl Instance Pointer (mip). This is a static global pointer that gives access to all the internal workings of MIRACL:
\begin{verbatim}
miracl *mip = mirsys(5000,0);
\end{verbatim}

In C++, this can be replaced by the creation of a MIRACL instance and then taking its address, for example:
\begin{verbatim}
Miracl precision(5,0); 
miracl* mip = &precision;
\end{verbatim}

However, even this is not necessary in this particular project. The basis for the pairing computations we use is the class \cPFC (for Pairing Friendly Curve), declared in \verb|pairing_3.h|. For the choice of curve of this project, the respective implementation is in \verb|bn_pair.cpp|. There, it can be seen that the constructor of \cPFC already calls \code{mirsys}, and so another call is not needed. 

Therefore, in this project initialization is done simply by creating a \cPFC class and then getting a handle to the Miracl pointer. This can be done with:
\begin{verbatim}
PFC pfc(AES_SECURITY);  
miracl *mip=get_mip();  
mip->IOBASE=16;
\end{verbatim}

The third line is not strictly needed. It is already included in the constructor of \cPFC. I include it only for clarity, so that the reader will know its value without having to read the source code of an external library.

Class \cPFC does not redefine a copy constructor, and uses the default. This does not seem to work well in practice, attempts to pass an instance of \cPFC by value have always resulted in a long list of errors. There should be no reason to do that either, as this is mostly used for access to functions over external data. Therefore, variables of \cPFC type should always be passed as references, which is also the fastest way to pass values in and out of functions.

\subsection{\fhutils / \fcutils / \code{utils\_impl.tcc}}
\label{sec:utils}

These files hold declarations and definitions of basic constants and functions to be used throughout the whole code. In most cases, they are nothing more than conveniences.

The first such convenience is the full set of included headers for the whole project. The macros beginning with \code{sh} denote ANSI codes for colours, so that the program can write coloured to the terminal. However, when redirecting the output of the program to a file, these colour codes will remain untranslated and pollute the output. They are mainly used to help in quickly reporting to the user, mostly for console debugging and for communicating tests' results.

The macro \code{NODEBUG} controls whether the program should output debugging information. This is done by calling special functions \code{DEBUG} and \code{ENHDEBUG} in the code as convenient. If the \code{NODEBUG} is \emph{not defined}, then the debugging mode is active, and some text is output to the terminal. Otherwise, these instructions just do nothing. The difference between \code{DEBUG} and \code{ENHDEBUG} is strictly cosmetic, offering two different colour schemes to help the programmer and debugger distinguish information with different levels of importance.

Other macros \code{OUT}, \code{ENHOUT} and \code{REPORT} also serve to send coloured data to the screen, offering even more ways to distinguish information. These, however, are not turned off by the \code{NODEBUG} macro and so always produce a visible result.
All the debugging and reporting macros allow for a very convenient way to output data, by streaming directly to standard output without the need for conversion functions.

There are four error codes defined, their macros all starting by \verb|ERR_|. These are used in particular places where the code must throw an exception, because data input from the outside lead to some particular kind of error. The value of these macros is a \code{std::string}, that is meant to be output as the prefix of the error message of the thrown exception.

Following, are three constants, of type \code{std::string}, that tell the program how operators for policy expressions are represented. Currently, \code{OR} represents logical OR, \code{AND} represents logical AND and \code{THR} represents a threshold function. Policies for use with this project should be written in prefix, or functional, notation, with each policy operator followed by a parenthesized argument list. Arguments, that is, attributes, should be represented by positive integer numbers only.

Finally, these files declare and define a few functions. 

\begin{description}
\item[\code{guard}] this is basically an assertion: if the condition fails, the program stops. The difference to a normal \code{assert} is that this function outputs a message before stopping.
\item[\code{test\_diagnosis}] this function offers a standard way to run a test: it writes whether the test was successful, and if not increases an error (test failure) count.
\item[\code{print\_test\_diagnosis}] this function prints a final message reporting whether a test batch was successful or how many tests failed.
\item[\code{convertStrToInt}] this is a convenient function to parse a \code{std::string} into an int. If the \code{std::string} does not represent an integer number, the function throws an exception with the appropriate error message.
\item[\code{convertIntToStr}] this function calls a function from the standard library to do this conversion. It provides an alternative, possibly easier to remember, name.
\item[\code{exprTokenize}] this function is important for use in parsing policies. It divides a \code{std::string} into individual tokens, which are separated by a single delimiter that can be passed as argument. The result is returned in an outbound \code{std::vector}. Unlike a normal tokenizer, this function is able to recognize parts of the sub-expression that are protected by parentheses, and which are not tokenized. For example, tokenizing the expression \code{3, 7, g(1,5,2)} with a delimiter `,' will yield 3 tokens: \code{3}, \code{7}, \code{g(1,5,2)}.
\item[\code{trim}] this is a basic function that removes all white space from the beginning and the end of a \code{std::string}.
\item[\code{isSuffix}] this tests whether the second argument begins with the first argument. Both arguments are of type \cstring.The name is misleading, as indeed no argument is a suffix of the other, but instead the first is a prefix of the second, or the second is the first with an added suffix.
\end{description}

Finally, a group of template classes are declared, for use in several places of the project. Template classes must be defined in the header file, but in order to keep the distinction of declaration and definition in different files, I have adopted a standard practice: define template functions in a third file and include at the right point in the header file. This third file is \verb|utils_impl.tcc|.

\begin{description}
\item[\code{contains}] this function receives a vector with elements of type \code{T} and an element of said type, and returns the index of this element in the vector, if it exists there. Otherwise, it returns $-1$.
\item[\code{addVector}] this function receives two vectors of the same type and appends all the elements of the second vector to the first vector, in the same order.
\item[\code{debugVector}] this is a function that receives a vector of some primitive type and a \cstring message. It lists the contents of the vector, preceding each element by the message and its position. It requires that the vector type can be output to a stream. This relies on the \code{DEBUG} macro, and so it will not produce results if \code{NODEBUG} is defined.
\item[\code{debugVectorObj}] this is just like \verb|debugVector|, except it works with types that can not be streamed but implement a method \verb|to_string()|. 
\item[\code{outVector}] this works just like \verb|debugVector|, but instead uses the \verb|OUT| macro. Therefore, this always prints something (unless the vector is empty).
\end{description}

\subsection{\fhss / \fcss}

These files hold declarations and definitions for the classes \cST, \cAP and \cSS. The first is a basic class to represent the concept of a share in a secret sharing scheme, and has little functionality. The others are abstract classes, that can not be directly instantiated, and represent the concepts of Secret Sharing scheme and a corresponding Access Policy.

\paragraph{\cST}

This class represents a share. It maintains three pieces of information: 
\begin{itemize}
\item the share of a given secret, which is a \cBig value;
\item the participant that holds this share, which is an integer. In this project, participants of a secret sharing scheme, and consequently attributes of an ABE scheme, are represented by integers.
\item an identifier for the share, that makes it unique within a given policy. This is a \cstring and should hold all the public information necessary for the reconstruction process.
\end{itemize}

The methods of this class are mostly trivial, allowing access to the internal data, string representation of a query and the basic needs: constructors, copy-constructor, assignment operator and equality testing.  

\paragraph{\cAP}
\label{para:cAP}
This class represents an Access Policy. Since this is the base class for any such policy, it has only basic functionality. The only internal data is a list of participants, held in a \code{vector<int>}. Constructors at this level only initialize this participants list in three different ways: with one single participant, with participants numbered from $1$ to $n$, or with an arbitrary list of participants. 

It has the following non-trivial functions:
\begin{description}
\item[\code{findCoefficients()}] {\bf pure virtual} this function returns the reconstruction coefficients associated with a set of share identifiers of a linear secret sharing scheme. It is a property of such schemes that such a set of coefficients exists and is independent of the share values themselves.

The share identifiers do not have to be all different, and neither have they to correspond to all shares of the policy. Ideally, they should all be compatible with the policy, although checking for that is the responsibility of the concrete class implementing this function. Since the share identifiers contain all public information, this is enough to compute the coefficients.

This function is pure virtual in this project. However, it is possible to give a default implementation using the coefficient extraction algorithm presented in [Crampton,Pinto 14], which means that in absolute terms it should be virtual but not purely so. That algorithm is relatively efficient, but much slower than a dedicated algorithm, since it requires a linear number of share distributions and reconstructions. Keeping this function pure virtual at least forces any new secret sharing scheme to implement a proper dedicated method. 

\item[\code{getNumShares()}] {\bf pure virtual} this function returns the number of shares generated by this policy.
\item[\code{evaluateIDs()}] {\bf pure virtual} this function receives a list of share identifiers and decides whether they satisfy the policy. If so, it returns a minimal set of shares that allow reconstruction of the secret.
\item[\code{obtainCoveredFrags()}] {\bf pure virtual} this method receives a list of participants (\code{vector<int>}) and returns two lists of indices into key and ciphertext fragments (also \code{vector<int>}) and a list of identifiers of the corresponding shares.

Firstly, this function clearly intrudes in the concept domain of ABE, as it is aware that it must return indices into lists of fragments. The argument names reflect this position, as they suggest ``attributes'' instead of ``participants''. The reason it ended up in \cAP and not \cKPABE is that its function requires to verify all shares in the policy which presupposes knowledge of it.

A better design, which I did not have time to implement, would be to have instead a function in \cAP that returned a list of share identifiers and participants in a given order, and then extract \code{obtainCoveredFrags()} to \cKPABE, passing it that list of identifiers. As it is, though, each policy implements its own traversal directly and makes these checks on the fly which, at least, has the benefit of being faster.

Now, the purpose of this function is to use the first argument, with a list of participants, representing the attributes present in a ciphertext, and identify which shares match those participants (which key fragments correspond to attributes in this list). For each match, return in the outbound variables: the index of the share (it is assumed that key fragments submitted by a decryptor are in exactly the same order as the shares are traversed here), the index of the participant that matched (covered) the share, and the identifier of the share. All these vectors should therefore have the same size, and the data in the same indices should relate to the same share. 

This matching is needed to perform the pairing operation in the decryption process.

\item[\code{evaluate()}] this function receives a list of \cST instances and decides whether they satisfy the policy. The default implementation, present in \fcss, creates a new vector with the share identifiers of the received shares and calls \code{evaluateIDs()}. After receiving the result, it uses the witness share identifiers to select the witness shares to return to its environment.

Pure virtual functions are not defined in this class, and must forcibly be implemented by any \emph{concrete} class descendant of \cAP. They are so specific to the concrete implementation that it is impossible to give code for them at this level.
\end{description}

\paragraph{\cSS}

This class represents a generic secret sharing scheme. It is an abstract class, with only basic functionality. This class holds a pointer for one \cAP, a \cBig representing the order of the underlying groups and a reference to an external \cPFC instance. As explained in Section \ref{sec_miracl}, this is because there is no need, nor indeed a way, to copy instances of \cPFC.

The reason why this class holds a pointer, instead of an instance of \cAP, is the following: both \cSS and \cAP are abstract classes. Important methods will be called by their descendants, which will have a specific type descendant of \cAP. Therefore, descendants of \cSS need to downcast their internal policy instance. C++ affords a safe way to perform downcast, by using the operator \verb|dynamic_cast|. This, however, requires the argument to be a pointer or a reference. For added safety, I have used smart pointers instead of raw pointers. A positive consequence of this option is that the memory footprint of \cSS is smaller since it does not need to store the complete information of a \cAP. This also saves the time needed to copy instances.

A possible alternative would have been to represent the policy as a reference instead of a pointer. This would force the policy to be fully initialized in the constructor. This inflexibility might not be a good idea. In fact, I am not entirely happy with the present design: the benchmark code shows a scenario where it would have been preferable to initialize \cSS without defining the \cAP immediately and instead define this later via a specific method. That would have allowed the benchmark for key generation and decryption, for example, to setup the KPABE scheme only once, defining the randomness and the attributes, and then changing the policy for the several scenarios. Currently, the only way to specify a new policy is to create the instance a new, and this entails doing the setup yet again. This is a big hit in the running time of the tests, although it does not affect the measured performances for key generation and decryption, since the timers only start after the setup has been done.

Changing the code to accommodate that change requires code to check that the KPABE class can not function while it does not have a policy and possibly adding a copy constructor in each \cSS class and maybe its descendants. There might be more elegant solutions, but lack of time prevented me of exploring them. 


The non-trivial functions are:
\begin{description}
\item[\code{getSharesForParticipants()}] {\bf static} this function receives a vector of shares and a vector of participant numbers, and it returns a new vector that contains exactly those shares whose participant is contained in the first argument.
\item[\code{distribute\_random()}] {\bf pure virtual} this function implements the distribution algorithm of the secret sharing scheme. This is a randomized algorithm, but the details of how to generate that randomness are the responsibility of the descendant classes. 
\item[\code{distribute\_determ()}] {\bf pure virtual} in some purposes, for example for tests, it might be necessary to perform the distribution with fixed random coins. That is achieved by this method, the deterministic version of \verb|distribute_random()|, which receives the randomness as an argument.
\item[\code{getDistribRandomness()}] this method simply returns the last randomness that was used to perform a distribution.
\item[\code{reconstruct()}] this method receives a list of shares and uses its internal policy to interpret them accordingly and reconstruct the secret.
\end{description}


\subsection{\fhkpabe / \fckpabe}

These files hold the declarations and definitions for the class \cKPABE. This represents the concept of Key-Policy Attribute Based Encryption scheme. Its internal data include:
\begin{description}
\item[\code{m\_scheme}] a smart pointer for a secret sharing scheme;
\item[\code{m\_pfc}] a reference to a \cPFC instance;
\item[\code{m\_nAttr}] the number of attributes in the universe;
\item[\code{m\_privateKeyRand}] the random value that is split among the attributes during the key generation. This is written in [Crampton,Pinto'14] as $t$;
\item[\code{m\_lastCTRandomness}] the random value that was used to blind the ciphertext in the last encryption. This is written in [Crampton,Pinto'14] as $s$;
\item[\code{m\_order}] the order of the underlying groups;
\item[\code{m\_privateAttributes}] the secret values associated to each attribute. These are written in [Crampton,Pinto'14] as $t_a$;
\item[\code{m\_publicAtts}] the public values associated to each attribute. These are written in [Crampton,Pinto'14] as $T_a$. In this paper, the public attributes are computed as elements of $G_2$, but the present code allows the choice between $G_2$ and $G_1$. That is determined by which macro is defined, either \verb|AttOnG1_KeyOnG2| or \verb|AttOnG2_KeyOnG1|. The latter corresponds to the scheme described in the paper.
\item[\code{m\_P, m\_Q}] the generators for groups $G_1$ and $G_2$ respectively;
\item[\code[m\_publicCTBlinder] the factor that is multiplied by the message in order to produce the ciphertext. This corresponds to the pairing of group generators raised to \code{m\_privateKeyRand};
\end{description}

The public interface of this class has several non-trivial functions.
\begin{description}
\item[\code{paramsgen()}] this function initializes the values of the group generators: \code{P} for $G_1$ and \code{Q} for $G_2$. These are passed from the outside as outbound arguments. It also computes the pairing of these generators and fixes the order of the ABE scheme to be that of the \cPFC instance.
\item[\code{setup()}] this function creates the key randomness $t$ and ensures it is smaller than the group order. It also randomly generates all the private attribute values, $t_a$. Then, it computes the public ciphertext blinder, $e(P,Q)^t$, and the public attributes, either $P^{t_a}$ or $Q^{t_a}$.

Notice that the key randomness is biased: if the group order does not evenly divide the possible largest value for a \cBig that \code{PFC.random(Big)} can generate, then some values will be more frequent than others. For a small example, consider that random would uniformly generate values from $0$ to $17$ inclusive, and that the order would be $11$. Then, final values from $0$ to $6$ would be twice as frequent as values from $7$ to $10$. In production code, a better way of generating randomness should be used.

\item[\code{genkey()}] this function generates a decryption key, that is, a set of key fragments. First, it invokes the distribution algorithm of the secret sharing scheme, and then calls \code{makeKeyFrags} on the resulting shares (see paragraph \ref{para:private}). 

Note: If it receives a vector of randomness as argument, it calls the deterministic version of the distribution algorithm.

\item[\code{encrypt(), encryptS()}] these functions perform an encryption. The main work is done by calling the private function \verb|encrypt_main_body|, which returns a blinder and a set of ciphertext fragments. This blinder is then combined with the message to produce the full cipherext in either of two ways (see paragraph \ref{para:encrypt}).
\item[\code{decrypt(), decryptS()}] these functions perform a decryption. The main work is done by calling the private function \verb|decrypt_main_body|, which returns a blinder if decryption is successful. This is then combined in the appropriate manner with part of the ciphertext in order to produce the original message.
\end{description}

\paragraph{Choice of group for fragments and public attributes}

As has been mentioned above, the project allows two ways to define key fragments, depending on which group is used to build key fragments, and which group is used to build public attributes and ciphertext fragments. The choice depends on the definition of one constant, and if the compilation commands are used, they are never both set at the same time.

To avoid code duplication, the code is the same for both variants, except that the types of certain structures change. In these cases, type declarations and function signatures are compiled conditionally, inside \verb|#ifdef...#endif| blocks. This happens frequently in these files, making the code a bit hard to read.

\paragraph{Alternate encryption and decryption}
\label{para:encrypt}
The \cKPABE class offers two ways to do encryption and decryption. One implements exactly what is in the papers in the literature: a message is an element of $G_T$, and the ciphertext includes its multiplication by a public blinder factor. The second way is taken from the MIRACL implementation of Fuzzy-IBE: a message is a string, and the ciphertext includes the EXOR of this string with the hash of the public blinder factor. 
Most of the work done in either case is exactly the same, and has been extracted into these auxiliary functions \code{encrypt\_main\_body} and \code{decrypt\_main\_body}.

In both cases, encryption works by first computing a blinder factor that raises the pairing $e(P,Q)$ to the key and the ciphertext randomnesses. The two methods then differ in how this blinder is used: in one case, it is multiplied with the plaintext message; in the other, it is hashed into a string and the result is then EXORed with the plaintext. Since there was no indication in the MIRACL source code why this alternative was used, I coded both and hope that the benchmark testing can show a justification in terms of performance.

\paragraph{Private functions}
\label{para:private}
There a few functions that are strictly  auxiliary, and need not be part of the exposed interface of the class. These are:

\begin{description}
\item[\code{makeKeyFrags}] this function receives a list of shares and produces a corresponding list of fragments, which are elements of either $G_1$ or $G_2$.
\item[\code{encrypt\_main\_body}] this function implements the essential part of the encryption process, that is equal for both encryption alternatives. It obtains fresh randomness and then computes the public blinder. It also creates the ciphertext fragments, by masking each attribute's public value with the ciphertext randomness.

\item[\code{decrypt\_main\_body}] this function is the counterpart of the previous one, and implements all the essentials of decryption. It first invokes its policy's \code{obtainCoveredFrags} to obtain a matching between the fragments in the decryption key and those in the ciphertext. If these are enough to satisfy the key's policy, then it proceeds with decryption.

It first obtains the reconstruction coefficients for a set of fragments that is sufficient and minimal to decrypt the ciphertext, by calling \code{findCoefficients}. Instead of performing an individual pairing between each couple of fragments in this set, we follow the technique in MIRACL's implementation of Fuzzy-IBE. We store in arrays of elements of $G_1$ and $G_2$ the couples of elements to be paired, and then execute the \cPFC function \verb|multi\_pairing| that realizes all the pairings at once, in a more efficient manner.

To avoid the exponentiation of an element in $G_T$ by a reconstruction coefficient, the latter is multiplied by the attribute fragment, which by linearity of the pairing will accomplish the same result. The choice of doing this multiplication always over the attribute fragments was arbitrary: it will be an exponentiation over $G_1$ or $G_2$ and will afford us a useful comparison, in benchmark testing, between the efficiency of said operations. In production code, this multiplication should always be done on the most efficient group, which means that in one case it should be done over attribute fragments and in other cases over key fragments.

Since \verb|multi\_pairing| receives two arrays of pointers to $G_1$ and $G_2$, all their elements must be valid addresses. For that reason, when the fragment to be passed into the array is first multiplied by a coefficient we have to store it in a third array, to force it to be stored in memory and have a physical address. Otherwise, the compiler will produce a temporary value that then does not have a physical address leading to run-time errors.
\end{description}

\paragraph{Precomputation of elements}
The implementation of \cKPABE makes use of pre-computation, to make some operations over the groups $G_1$ and $G_2$ faster. This was inspired in such use of precomputation in MIRACL's implementation of Fuzzy-IBE. The latter did not give much intuition on when such precomputation can or should be used, so I only mirrored their use as much as possible. In short, this is what happens:
\begin{itemize}
\item in parameter generation for the scheme, precomputation is set for the generator of the group where the key fragments will be formed. This will make multiplications with $Q$ faster, and thereby the production of key fragments.
\item in the scheme setup, precomputation is set for the public attribute values. These values only depend on the private attribute values and on the appropriate group generator, and so are fixed at setup time. The attribute fragments are then produced by multiplying these public values by each ciphertext randomness, and so setting precomputation on each public attribute value will make the production of ciphertext fragments faster.
\item MIRACL also offers precomputation for pairing, but only over elements of $G_2$ (at least, for Barreto-Naehrig curves). This is set in two distinct ways in the current project:
\begin{itemize}
\item in \code{makeKeyFrags}, precomputation is set for each key fragment if these are built in $G_2$.
\item in \verb|encrypt_main_body|, precomputation is set for attribute fragments if these are build in $G_2$. 
\end{itemize}
Thus, in each pairing, it is the case that one fragment has precomputation set on it, to improve the efficiency of the operation.
\end{itemize}


\subsection{\fhblcanon / \fcblcanon}

These files hold the declarations and definitions for the class \cBLcanonAP and \cBLcanon. These represent a Benaloh-Leichter Access Policy represented in canonical form, that is, corresponding to a DNF formula, and the corresponding type of Benaloh-Leichter Secret Sharing scheme. They are, respectively, concrete descendants of the abstract classes \cAP and \cSS.

\paragraph{\cBLcanonAP}

This class represents an access policy in canonical form. Such forms consist of a disjunction of conjunctive clauses, each of them representing a minimal set. This makes for a very simple data structure, and accordingly simple-ish code.

The internal data of this class include a description of the represented policy, which is a \cstring, and a vector of minimal sets, each represented as a vector of integers (\code{vector<int>}). The resulting type is \code{vector< vector<int> >}.

Each share of the policy is uniquely identified by a \cstring with format \code{nn:mm}, where \code{nn} is an integer numbering the minimal set the share belongs to (starting at $1$) and \code{mm} is an integer representing the share participant (which normally should not be $0$). 


NOTE: It is assumed that the same literal does not appear twice in each conjunctive clause, as that is logically redundant. This, however, is not checked, and could possibly be exploited by a malicious user. Some guard against this should be introduced in the code at some point.

There are the following non-trivial functions in the public interface:
\begin{description}
\item[\code{parseFromExpression()}] this is the function that interprets a policy and extracts the corresponding minimal sets. This function is recursive, due to the nature of a legal expression. A correct expression has two levels: at the highest level (0), it has one OR operator and a its arguments are AND expressions; at the lowest level (1), it has one AND operator and its arguments are numbers. The first call of this method should be at level $0$, and eventually it will call itself several times at level (1). 

In each call, it first verifies extreme cases: if the policy is empty there is nothing to do, and if it is only one number, then it creates a single minimal set with a single element. The general case is handled in the \code{catch} block of the \code{try...catch} construction. This is triggered if the conversion from the expression into an integer was not possible, which means the expression is some richer string. At a high level, this identifies the operator of the expression, the argument list and then tries to parse each argument to produce the internal representation. More detailedly:
\begin{itemize}
\item the function tries to locate a legal operator at the start of the expression by locating the first instance of an opening parenthesis.
\item if the function does not find the right operator at the level it is in, it throws an exception and finishes.
\item after locating the first opening parenthesis, the function looks for the last closing parenthesis and checks that this is the last character in the expression. 
\item it retrieves the expression inside the two parentheses and identifies the individual arguments in it, using \code{exprTokenize} (see \ref{sec:utils}).
\item if it is at level $0$, it makes a recursive call for each of the identified tokens, processing them now at level $1$. The result of each process is a minimal set, so it collects them in a vector of minimal sets and returns the result.
\item if it is at level $1$, then it tries to interpret each of the tokens as a participant and to produce the corresponding minimal set for the  expression. 
\end{itemize}


\item[\code{getNumShares()}] in a Benaloh-Leichter scheme, the number of shares is the sum of number of elements in each minimal set. 
\item[\code{evaluteIDs()}] this is an implementation of a virtual method in the parent function. It returns true if the vector of shares it receives as argument satisfies any of the minimal sets of the policy. It returns as soon as it finds a satisfied minimal set and so produces a minimal list of witness shares.
If the current minimal set is not satisfied, it ensures that the vector \code{satisfyingSharesIndices}, which may have been altered in the call to \code{satisfyMinimalSet}, is cleared.

\item[\code{findCoefficients()}] this function returns the reconstruction coefficients for a vector of shares. In the Benaloh-Leichter scheme, these are always $1$ or $0$. Since this function is supposed to be called for minimal sets, the result is always $1$. Otherwise, this function would have to reproduce work that is done elsewhere to find a minimal set of the shares it receives, which would be more expensive. Therefore, this method makes no checks and assumes all shares are needed for the reconstruction of the scheme and so all of them have coefficient $1$.

NOTE: Currently, this function has redundant code, but I can not change it because I am midway through benchmarking. In short, the conditional to check if \code{id} is in \code{shareIDs} is obviously unnecessary, since \code{id} was defined as exactly one certain element of that vector. The inside of the \code{for} loop should be only \code{coeffs.push\_back(1);}.

\item[\code{obtainCoveredFrags()}] this function discovers which shares are covered by the attributes received as argument. It processes each minimal set in the policy individually. For each element in this set:
\begin{itemize}
\item it obtains its share identifier
\item checks if this identifier is in the inbound vector
\item if it is, it updates the outbound structures (see documentation for parent class)
\item the variable \code{count} keeps track of how many shares have been inspected so far, across all minimal sets.
\end{itemize}
\end{description}

\paragraph{Private and protected functions}
\begin{description}
\item[\code{init()}] this function is called in the constructors, and calls \code{parseFromExpression()} to compute the minimal sets that compose the 
initialize the internal representation of the policy. 
\item[\code{satisfyMinimalSet()}] this is a protected function that is called by \code{evaluateIDs}. It receives a minimal set (\code{vector<int>}) and a list of shares identifiers, and verifies if each element in the minimal set is satisfied by at least some share. If the test is successful, the function also returns in an outbound parameter the indices of the shares (in the inbound vector argument) that satisfy the elements of the minimal set. 

This function checks each element of the minimal set. It computes its share identifier and checks if it is in the vector of shares. If it is, it adds the index of the satisfying share to the outbound argument. If it isn't, it leaves the outer \code{for} loop and clears the outbound vector, before retuning \code{false}.
\end{description}

\paragraph{\cSS}

This class represents a Benaloh-Leichter secret sharing scheme equipped with a canonical policy. Its internal data include a pointer to a \cBLcanonAP, which is only a convenient casting of the member \code{m\_policy} of the parent class. This recasting is done in the function \code{initPolicy} during the constructor call.  They also include a vector of \cBig that represents the randomness used in the distribution function.

There are the following non-trivial functions:

\begin{description}
\item[\code{manageRandomness}] this function is used with two purposes, indicated by a \code{RandomnessAction} argument: initialization or randomization. This is an enumeration defined as a protected member of \cSS, which has two possible values: \code{init} and \code{randomize}.

For each minimal set, this scheme requires a number of random pieces equal to the size of the minimal set minus one. The function traverses the internal structure of minimal sets linearly, and executes the correct number of times for each set. Each times this inner loop executes, it checks the action given as argument and:
\begin{itemize}
\item if it is \code{init}, it createa a new position in the randomness vector, with value $0$
\item if it is \code{randomize}, it invokes \code{PFC.random(Big)} to randomize the respective position in the randomness vector. This is so because \cPFC does not offer a function to produce a new random value, it can only randomize a reference it receives as argument.
\end{itemize}


\item[\code{distribute\_random()}] this calls \code{manageRandomness} to obtain fresh randomness and then calls \code{distribute\_determ} with it.
\item[\code{distribute\_determ()}] this distributes the secret among each minimal set. For each set, each share but the last is a different random value, taken from the randomness vector. The last share is the secret minus the sum of all these values. After the right value of the share is computed, it is used to create a \cST element that is then added to the output vector.

\item[\code{reconstruct()}] this function reconstructs the secret from a set of shares. It first calls \code{evaluate} to obtain a minimal set of shares if a successful reconstruction can be made. If successful, it simply sums the value of all the shares in the minimal set.
\end{description}

\subsection{\fhtree / \fctree}

These files hold the declarations and definitions for the implementation of a tree data structure. This structure is implemented via classes \cnode and \ctree, which are specific to the specific problem of representing a tree-like Secret Sharing Access Policy.

Access Policies can, in general, be represented by a parse tree, that corresponds to a functional view of the policy expression. A Shamir Tree is a good example to understand this: the whole policy can be seen as a boolean function, defined by a threshold operator that takes as arguments the result of evaluating other threshold operators. This naturally leads to a tree like structure where each branch is evaluated fully before the final expression can be computed. Given the focus of this project, this tree is not agnostic of its contents, and knows what kind of data it holds and what variants are allowed. It is meant to work not only with Shamir tree structures, that only require threshold gates, but also with, for example, a generic Benaloh Leichter scheme that would allow OR and AND gates on top of the Threshold ones. It is naturally possible to extend this tree to accept other kinds of gates and accept, eventually, generic circuits as policies.

File \fhtree defines two enumerations that are of fundamental importance for these and the Shamir Tree classes:
\begin{description}
\item[\code{NodeContentType}] this represents the three kinds of node that the tree may have: leaf nodes, inner (non-leaf) nodes, and an empty node, that is merely a placeholder until something useful comes along.
\item[\code{InnerNodeType}] this represents the three different kinds of non-leaf node, according to the corresponding policy operator: AND, OR and THRESHOLD.
\end{description}

\paragraph{\cnode}
Class \cnode represents the contents of an individual tree node, independently of its placement inside a tree. There are different kinds of \cnode.

NOTE:  In hindsight, it would have been better to implement them as individual classes, all descendent of \cnode. Yet, the decision was made at the start that the nodes would be distinguished by their type, as it seemed that there was not much behaviour differing according to node type.

The internal data of a node include:
\begin{description}
\item[\code{m\_type}] a variable of type \code{NodeContentType}, indicating the type of this node
\item[\code{m\_innerNode}] a union that describes the node in more detail. Since the node can be of three different types, it may have different data describing it:
\begin{description}
\item[If it is a leaf node] it only stores its value in \code{m\_leafValue}
\item[If it is an inner node] it has a sub-type (a variable of type \code{InnerNodeType}) and two numeric values, that are enough to characterize any of its three possible sub-types. AND and OR gates only have an -arity, that is, the number of arguments they take; THRESHOLD gates add to this the value of their threshold.
\item[If it is a nil node] none of these data should be used. This has to be enforced by the functions that use this data, which once again is a clear sign these should have been individual classes.
\end{description}
\end{description}

There are the following non-trivial functions:
\begin{description}
\item[Accessor functions] These functions would ordinarily be considered trivial, since they only return interal data. However, given the fact that the different types of nodes are handled via member data and not an object-oriented design of classes and polimorphism, these functions become more complicated to ensure that they are only called for the correct types of nodes. Accordingly, they all start with a test of the node's type and return an exception in case that is not correct. Otherwise, they simply return the right member data.
	\begin{description}
	\item[\code{getLeafValue()}] 
	\item[\code{getInnerNodeType()}] 
	\item[\code{getThreshold()}] 
	\item[\code{getArity()}] 
	\end{description}
\item[\code{to\_string()}] this function returns a \cstring description of the contents of the node. This indicates in a convenient format the parameters that govern it.
\end{description}

\paragraph{Node Construction}

Uncommonly, this class does not have a public constructor. All the constructors it defines are private. That means that it is not possible to directly create an instance of \cnode. Instead, it implements the Factory Method software pattern. The reason for this is again the handling of different types through members instead of classes. Giving a public constructor would make it possible to create inconsistent instanes of \cnode, with internal data mismatching the node type. By only offering factory methods instead of a constructor in the public interface, it is possible to exactly regulate how instances of \cnode are created and ensure that they are all internally consistent. These methods are all static, since there could not be an instance of \cnode on which to call these methods. 

All these methods return pointers (actually, smart pointers) to \cnode, instead of direct instances, since the tree structure needs to work with pointers. Otherwise, obtaining nodes from the tree and changing them such the changes are reflected on the tree will involve tedious and wasteful copying of structures. 

The \code{shared\_ptr} pointers that these methods return can not be initialized by the preferred way of invoking \code{make\_shared}. That is because \code{make\_shared} would implicitly call the appropriate constructor, but since these are private, that call would fail. Instead, these pointers are initialized from a raw pointer to \cnode.

List of methods:

\begin{description}
\item[\code{makeNILNode()}] returns a pointer to a \cnode of \code{nil} type. Does not initialize any internal data.
\item[\code{makeOrNode()}] returns a pointer to a \cnode of \code{inner} type and \code{OR} sub-type. It initializes the inner type and the first argument of the inner node's data.
\item[\code{makeAndNode()}] returns a pointer to a \cnode of \code{inner} type and \code{AND} sub-type. It initializes the inner type and the first argument of the inner node's data. 
\item[\code{makeThreshNode()}] returns a pointer to a \cnode of \code{inner} type and \code{THR} sub-type. It initializes the inner type and both arguments of the inner node's data.
\item[\code{makeLeafNode()}] returns a pointer to a \cnode of \code{leaf} type. It initializes the leaf value.
\end{description}

\paragraph{\ctree}

Class \ctree represents nodes of a tree. Each instance of \ctree should be understood as a \cnode \emph{in a tree}, since it keeps its relation to other instances of \ctree. When taking together, these connected nodes represent the whole tree.

This class is not as dynamic as a general tree. It is expected that trees are formed once and then not changed again. Therefore, the content of nodes should not change. This means that a node that is going to rest at the bottom of a tree should be created as a leaf node even before it is added to any tree. Analogously, a future non-leaf node should be created as an inner node.

The tree assigns a unique identifier to each node. The ID of an inner node represents its position in the tree, encoding the path from the root to that node. The ID of root is simply \code{0}. The ID of the $i^{th}$ child of a node with ID \code{y}, where $y$ is some \cstring, is \code{y:i}, where $i$ is an integer that may start at $0$.

Leaf nodes have some extra information, because these IDs are used as the identifiers of the respective shares. These have to hold all the public information of the share, which includes its participant index. Thus, the Id of a leaf node is equal to the ID of an inner node in the same position followed by \code{:=n}, where $n$ is the participant number.

The internal data include:

\begin{description}
\item[\code{m\_node}] a pointer to a concrete \cnode, that holds all the information about this node of the tree;
\item[\code{m\_nodeID}] a \cstring that includes the unique identifier of this node in the tree;
\item[\code{m\_childen}] a vector of pointers to \ctree that represent the children nodes of this node;
\end{description}

NOTE: in what follows, when I write ``the current \ctree'' I mean the instance of \ctree whose function is executed.

This class has the following non-trivial functions:
\begin{description}
\item[\code{getChild()}] this method returns a pointer to the $i^{th}$ child of a node, if the index $i$ is legal. Otherwise, it throws an exception.

\item[\code{appendChild()}] this function takes a pointer to \cnode, creates a corresponding \ctree based on it and adds it to the current \ctree. The new node is inserted at the end of the current \ctree's children. In case the current \ctree is an empty tree (its own type is nil), then the inner \cnode of this \ctree is changed to be the new node and its ID is updated. If the current \ctree is a leaf node, then the operation is rejected, since leaf nodes can not have children. 

Otherwise, if the arity of the \ctree can accommodate one more node, the new \ctree is added to the children and given a new ID.
\item[\code{appendTree()}] this function takes a tree rooted at a given \ctree and adds it to the current \ctree. It does the same checks of \code{appendChild} but has a little more functionality, since now it is adding a whole tree instead of just one node. For that reason, when it creates a new entry in its children list it invokes the function \code{updateID} to change the IDs of all the nodes of the appended tree, so that these reflect their new position in a larger tree.

\item[\code{to\_string(), full\_to\_string()}] these methods return a recursive description of the tree, listing a node and a parenthesized list of its sub-trees. The second version adds the ID of each node to the output, giving more complete information for debugging purposes but being harder to read.

\item[\code{getNumLeaves()}] this function returns the number of leaves in the tree. It works recursively, traversing the tree in a depth-first manner.
\item[\code{updateID()}] this method is invoked for a particular \ctree that has been added to a parent \ctree. Its purpose is to give the current \ctree and all its descendants an ID that reflects their new position. It is therefore invoked recursively. In every invocation, it receives as arguments the ID of the parent for this whole tree and its position among its children.

The first node of the appended \ctree must have an ID of \code{0}, and subsequent nodes will hang below this. Therefore, they will have an ID starting with \code{0:}. When the root becomes a child of another tree, its own ID changes from \code{0} to some prefix describing its position, say \code{x}. Consequently, all descendants of this node must reflect this change in their IDs, and the initial \code{0} must be replaced by the new ID of their initial root, that is, \code{x}.

NOTE: This method only works correctly when the tree that is appended is not a subtree of any other tree, since it assumes that the ID of the first node is "0". However, it is not natural to add the same nodes to two trees at the same time, and since there is no function in this class to remove nodes from a tree, it should never occur that we would be appending a sub-tree of another tree.

\item[\code{findIDForNode()}] this function computes the right ID for a node in the tree, deriving it from the ID of the parent node and the node's position among the parent's children as was explained above. A nil node does not have an ID. 
\end{description}


\paragraph{Node Construction}

Just like \cnode, this class implements a factory-method software pattern. Its main constructor is private and instances are created via factory methods that are static and return pointers to \ctree instances. There is a public constructor that only returns an empty tree. There is also a private method, \code{initNodeID()}, that computes a unique identifier for each node when the node is created.

When a \ctree is created, it is disconnected from any other nodes, forming a tree with only one element. The function \code{initNodeID()} is called in the constructor, giving a default root ID to the node, that is, \code{0} or \code{0:=x} for some integer $x$. 


\subsection{\fhshtree / \fcshtree}

These files hold the declarations and definitions for the classes \cShTreeAP and \cShTree. These represent secret sharing access policies that are represented by Trees of Shamir threshold nodes, and the resulting secret sharing schemes.

\paragraph{\cShTreeAP}

This class represents an access policy based on Shamir threshold gates. In its simplest case, it is just like the original Shamir scheme, where a reconstruction is successful if it presents at least as many shares as its threshold. To obtain the general case, shares are replaced by other sub-trees, meaning that a reconstruction is successful if a given number of sub-policies are successful. This in turn will depend on their subtrees until eventually success depends on presenting a given share.

A natural way to represent such policies is as a tree where each node represents a threshold basic secret sharing scheme, and sub-policies are indeed subtrees. A node's children can be either other trees or leaf nodes, representing individual shares. The value of a leaf is merely the participant associated to the corresponding share. 
This kind of policy is implemented by a tree based on the \ctree class.

The internal data of \cShTreeAP include \code{m\_description}, a \cstring that holds the expression representing this policy, and \code{m\_treePolicy}, a pointer to \ctree that represents the root of the tree for the policy. Despite the fact that the internal representation is based only on Threshold gates, the policy expression can be described using also AND and OR operators, since these can be converted to threshold representations and are convenient and better-known operators.

The class has the following non-trivial non-static functions in its public interface.
\begin{description}
\item[\code{parsePolicy()}] this is a redundant method that unwittingly escaped deletion before I started the benchmarks.

\item[\code{parseTreeFromExpression()}] this is a recursive method that receives an expression and returns a pointer to the root of the corresponding tree. 

Base cases:
\begin{itemize}
\item if the expression is the null string, it returns an empty tree.
\item if the expression can be converted into a number, it represents a leaf. The function creates the corresponding \cnode and transforms into a \ctree, which is then returned.
\end{itemize}

Recursive case:

If the conversion to an integer failed, which is caught in the \code{catch} block, the expression is more complex than a single number and must be fully parsed. This code is similar to a corresponding method in class \cBLcanonAP, except that it must handle more complex policies, with threshold gates and an unlimited amount of levels. The work occurs in several steps:
\begin{itemize}
\item As in \cBLcanonAP, it starts by locating an operator and identifying its argument list;
\item it tokenizes this list and creates a pointer to \cnode (\code{newNode}) with a type suiting the operator;
\item it creates a \ctree out of \code{newNode}, \code{pTree}.
\item if the operator is not a threshold gate, every token is a sub-expression, and so the function invokes itself recursively for each child. After the invocation, it appends the resulting tree to its own \code{pTree}.
\item if the operator is a threshold gate, the first token must indicate the threshold, and so the function proceeds as in the previous case for all tokens except the first.
\end{itemize}

It finally returns the tree that has thus been constructed.
 
\item[\code{findCoefficients()}] this function computes a vector of reconstruction coefficients, one for each share in the argument it receives. It is assumed that these shares can successfully recover the secret.

This is a fairly complex method that requires a few auxiliary methods. The reconstructing shares are not necessarily a full set of shares. If we trace the path to root from all the reconstructing shares, we expect to obtain only a subset of the policy tree. It is only this subtree that matters to this function. For reference, I call it reconstruction-tree.

Each node in the reconstruction tree has siblings and is associated to a Lagrange coefficient. The secret is obtained from computing the linear combination of the root's direct children's shares with their Lagrange coefficients in the reconstruction-tree. But these shares are themselves reconstructed from lower children, such that at each level we have to compute the appropriate Lagrange coefficients and combine them.

Due to linearity, this reduces to multiplying only the shares at the base of the tree by their full Lagrange coefficients. By this term, I mean the coefficients that result from multiplying all the coefficents in the path from a share to root. These coefficients may be used by different shares, since some paths may overlap at higher levels of the tree.

The strategy to find these coefficients is this:
\begin{itemize}
\item the function creates two maps (hashtables) that associate nodes in the tree to reconstructing sets and coefficients. Reconstructing sets are the child numbers of paths in the reconstruction-tree that arrive at that node, coming from the leaves. These are stored in \code{setChildNos}. Each element in such a reconstruction set has a different coefficient, and these will be stored in map \code{setCoeffs}.
\item the first loop obtains the prefix of each reconstructing share and invokes \code{storeSharePrefixes}, registering the path for this share in \code{setChildNos}.
\item a second loop traverses \code{setChildNos}. Each key in this map represents a different note in the reconstruction-tree. The corresponding value is the set of child numbers of the reconstructing elements. An inner loop computes the Lagrange coefficient for each element in such set and stores it in a vector \code{vec\_coeffs}. This is then added to the map \code{setCoeffs} with the key of the corresponding node. 
\item at this point the whole reconstruction tree is represented in the two maps, and the Lagrange coefficients of each node are computed. It remains only to compute the final coefficients for each share. This is done by a final loop that takes each share in the inbound argument for this method and invokes \code{findFinalCoefficient} for this share and the two maps. The resulting coefficient is then added to a result vector \code{coeffs}. When the loop terminates, this vector is returned.
\end{itemize}


\item[\code{getNumShares()}] this function returns the number of shares in the policy, which is equal to the number of leaves in the tree;
\item[\code{evaluateIDs()}] this function calls the recursive method \code{satisfyNodeID()} giving it the policy tree as a starting point. If this tree is satisfied, the policy is satisfied and this function simply relays the results of \code{satisfyNodeID()}.
\item[\code{obtainCoveredFrags()}] this function must be implemented because it corresponds to a pure virtual function in the base class. It calls the recursive function \code{obtainCoveredFragsRec()} with \code{count} initialized to $0$ and starting at the class's internal policy tree root.
\end{description}

\paragraph{Static functions}
It also has a number of static functions. All of these can be invoked without needing an instance of \cShTreeAP. They perform somewhat utility functions that are needed by the normal methods of the class.

\begin{description}
\item[\code{satisfyNode()}] this function receives a pointer to a \ctree and decides whether this is satisfied. If that is the case, it returns a minimal set of shares that satisfy the policy. It delegates the work to function \code{satisfyNodeID()}, that only receives a vector of share identifiers instead of a vector of \cST. This function, then, acts as a wrapper for this call, preparing the input and extracting its output from the result it receives. To prepare the input, it creates a vector with the identifiers of all the shares it receives; to extract an output, it creates a sub-vector of its shares whose index is in the vector returned by \code{satisfyNodeID()}.

\item[\code{satisfyNodeID()}] this function does all the work of verifying if a policy is satisfied. It calls itself recursively descending in the original \ctree until a direct satisfaction decision can be made at a base case.

It returns a boolean indicating satisfaction and the indices to a minimal set of sufficient shares via its outbound argument \code{satisfyingSharesIndices}, which I call the witness set. Each time this method finds a satisfying branch, it adds the contents of the branch's witnesses to its own witnesses. However, if at the end the satisfied branches are insufficient to satisfy the threshold, the current witness vector is erased: since the node is unsatisfied, there can be not be a set of satisfying shares.

The base cases for the recursion are:
\begin{itemize}
\item the \ctree is empty (nil): These trees are always unsatisfied by convention.
\item the \ctree is a leaf: such tree is satisfied only if its identifier is in the received vector of identifiers. If that is the case, the share's index is added to the witness vector \code{satisfyingSharesIndices} and the function returns success. Otherwise, the witness vector is cleared.
\end{itemize}

The recursive case happens if none of the base cases is satisfied. The procedure depends on the sub-type of the \ctree's internal node:
\begin{enumerate}
\item if it is an AND gate, it recursively invokes \code{satisfyNodeID()} for each child. If an invocation passes, the returned indices are added to the current witness vector. As soon as an invocation fails, this witness vector is cleared, and the function returns false;
\item if it is an OR gate, the logic is the dual. The function recursively invokes \code{satisfyNodeID()} for each child. As soon as one passes, the returned indices are added to the witness vector and the function returns successfully. If the loop ends, no child succeeded and so the function returns false;
\item if it is a Threshold gate, the function keeps track of the number of satisfied children so far, then recursively invokes \code{satisfyNodeID()} for each child: each time a child is satisfied, the indices are added to the witness vector. As soon as enough children are satisfied (the threshold number), the function returns true. If the loop terminates and the number of satisfied children is insufficient, the function returns false. 
\end{enumerate}


\item[\code{storeSharePrefixes()}] this function is used as an auxiliary for the computation of the share coefficients. Its task is to register a share's path in the map \code{setChildNos}. It calls \code{extractPrefixAndNoFromID} repeatedly to obtain each step in the path of the share to root. It bears noting that a share identifier should start in $0$ and that each number following a ``:'' is in fact the child number the path takes at each step. The loop traverses the path from the root to the leaf, and at each stage the code reconstructs the identifier of each node. This is used to check if there is already a set for that node in \code{setChildNos}. If there is no such set, one is created with the current path's number as sole element. If the set exists, this path number is added to it only if it is not already there. 

\item[\code{extractPrefixAndNoFromID()}] this function receives a share identifier and extracts the information about the share's parent, removing one level from its path to root: it returns in \code{prefix} the position of the parent's node; in \code{childNo} the current node's position among the parent's children; and removes the prefix from \code{shareID}. For example, the identifier \code{0:1:7:4:=3} is transformed into \code{1:7:4:=3}, returning $0$ in \code{prefix} and $1$ in \code{childNo}. In case the identifier describes a share with an empty path, for example \code{4:=3}, it does not change any outbound argument and returns false.

\item[\code{findFinalCoefficient()}] this function computes the reconstruction coefficient for a share in this policy. This requires building the Lagrange coefficient for its node in the path from the root to the share. This method is called as part of the \code{findCoefficients} method, and relies on receiving two maps (typically implemented as Hash tables) that hold crucial information about the tree. Both maps are indexed by node identifiers. To understand how these maps are build see \code{findCoefficients} and \code{storeSharePrefixes}.
\begin{itemize}
\item \code{setChildNos} associates a node to the child numbers of the relevant children. 
\item \code{setChildNos} associates a node to a set of Lagrange coefficients, one for each relevant child.
\end{itemize}

This function invokes \code{extractPrefixAndNoFromID()} to recover the path from the root to the share. While this function returns true, it is possible to cover one more step of that path and reconstruct the identifier of the respective node and the position of the path in that node. This position gives the right Lagrange coefficient for the share at this level in the tree. The coefficients for the share at each level are multiplied together and the result is returned.

\item[\code{computeLagrangeCoefficient()}] this function computes the Lagrange coefficient for a share within a set of shares. In fact, it delegates the work to \code{computeLagrangeCoefficientChildNos()} by first creating a vector with the child numbers of each share. These are extracted from the share identifier by the method \code{extractChildNoFromID}.

\item[\code{computeLagrangeCoefficientChildNos()}] this function computes the Lagrange coefficient for a share index given a set of share positions. The child number is just an index in a vector, and so it is 0-based. Participant indices for the computation of the Lagrange coefficient are never $0$, and so the method \code{extractPublicInfoFromChildNo} is used to obtain the right value of the participant from its child number. 

\item[\code{extractChildNoFromID()}] this method receives a share identifier and locates within it the child number of this share among the children of its parent. This is the value immediately before \code{:=}.
\item[\code{extractPublicInfoFromChildNo()}] this simply returns the child number plus one.
\end{description}


\paragraph{Private and Protected functions}

There is a private function \code{init} that is invoked only in the constructor to finish initialization. It only calls \code{parsePolicy()} to initialize \code{m\_treePolicy}. 

There is also a protected function \code{obtainCoveredFragsRec} that is called by  \code{obtainCoveredFrags} and start a recursive implementation of this virtual function of its ancestor class. The goal of this function is to identify which shares are covered by the fragments that are received as argument. Given the tree nature of the data structure, a recursive function is the easiest approach for the solution. 

There are three outbound arguments, all of them vectors that will have an entry per each covered share: \code{keyFragIndices} indicates the number of that share in the policy, which amounts to its index in a prefix enumeration of all leaves in the tree; \code{attFragIndices} will hold the index, in \code{atts}, of the attribute that covered the share; finally, \code{coveredShareIDs} will hold the ID of the covered share. The variable \code{count} must keep track of a leaf's position in the full tree, even when the function is invoked for a sub-case. This is done by passing \code{count} by reference, letting it serve as an inbound/outbound argument. Each time a leaf is found, this value is increased.

As base cases:
\begin{itemize}
\item it checks if the current node is empty (nil) and simply returns an empty result in that case, that is, it does not insert any data in the outbound arguments: \code{attFragIndices, keyFragIndices, coveredShareIDs}. 
\item it checks if the current node is a leaf. In that case, it only has to check if the participant associated to that share is in the received list of participants (this is named \code{atts}\footnote{For a comment on this name mismatch, see Section \ref{para:cAP}.}). If yes, then it adds the corresponding data to the outbound arguments.
\item in the remaining case, the node must still have subtrees under it, so it successively recursively invokes the function again for each of its children. 
\end{itemize} 


\paragraph{\cShTree}

This class represents a Secret Sharing scheme equipped with a Shamir Tree access policy. It is a concrete implementation of the class \cSS and so gives implementations for the pure virtual methods of that class.

Its internal data include a pointer to a \cShTreeAP,  which is only a convenient casting of the member \code{m\_policy} of the parent class. This recasting is done in the function \code{initPolicy} during the constructor call.  They also include a vector of \cBig that represents the randomness used in the distribution function.

There are the following non-trivial non-static functions in this class:
\begin{description}
\item[\code{distribute\_random()}] this performs a randomized secret distribution. To do this, it first calls \code{manageRandomness} to obtain fresh randomness and then calls \code{distribute\_determ} with it.
\item[\code{distribute\_determ()}] this distributes the secret among the policy, taking its randomness as an input. The implementation of the virtual method in the base class has to call a recursive version, due to the tree-nature of the policy.

The recursive version receives a tree indicating and has the mission of distributing a secret among this tree. It also receives an integer argument, \code{count}, indicating the first value of the randomness vector that it can use for this tree.
As a result, it produces a vector of \cST. If the tree is empty, it returns an empty vector, and if the tree is a leaf, it returns a simple share whose value is the secret.

The recursiveness is triggered for non-leaf nodes. These should all be of Threshold sub-type, and the function terminates with an exception if that is not the case. For an inner node, the function creates a polynomial of degree equal to the threshold, according to the scheme definition. The first value of the polynomial is the secret, and the remaining values are taken from the randomness vector. 

After the polynomial is thus initialized, it computes the share for each child of this node, evaluating the polynomial at the corresponding point. If this child is a leaf, then it creates a share and adds it to the result vector. If it is an inner node, then it descends recursively to distribute this share value among the sub-tree starting in that child. Since this sub-tree also consumes randomness, the changes to the value of \code{count} inside this invocation have to be reflected for the caller, and that is why this variable should be passed by reference.


\item[\code{reconstruct()}] this method reconstructs a secret from a set of shares. It first checks if those shares satisfy the policy, by invoking its \code{evaluate} method. If only one share is needed to satisfy the secret, then that share's value is the secret and so the function simply returns it. Otherwise, it creates a vector with the identifiers of those shares and calls \code{findCoefficients()} to obtain the general reconstruction coefficients for the shares. Finally, it computes the linear combination of shares and coefficients that returns the secret.
\end{description}

\paragraph{Protected Functions}

There are a few protected functions in this class. The function \code{init} is called in the constructor to finish initialization. It does this by calling \code{initPolicy} and \code{manageRandomness}. 
\begin{itemize}
\item \code{initPolicy} initializes \code{i\_policy} by downcasting \code{m\_policy}, as was done in \cBLcanon.
\item \code{manageRandomness} covers exactly the same purpose of the equally named method in \cBLcanon. Unlike that class, the version of \cShTree requires the use of a recursive version that is invoked by the simple one.

This scheme requires a number of random pieces for each simple threshold node in the tree. This number is equal to the threshold minus one. 
The recursive version receives as arguments, besides the action to perform, a pointer to a \ctree and an integer. The function traverses the tree it received as argument depth-first managing the randomness for each inner node it encounters. Leaf nodes are not associated to any randomness. The variable \code{count} is used for the \code{randomize} action, to keep track of the positions that have been already randomized before the current subtree, similarly to \code{obtainCoveredFragsRec()}.

\end{itemize}

\paragraph{Alternative reconstruction}
The reconstruction algorithm given above is the general reconstruction method proposed in Crampton, Pinto'14. It works for any linear secret sharing scheme. For the KPABE scheme, this implementation is not relevant, as the reconstruction algorithm is never called directly. Instead, the KPABE scheme uses the reconstruction coefficients returned by \code{findCoefficients}. But this class is a Secret Sharing class, and can be used outside the scope of a KPABE scheme. 

In this case, it is conceivable that a dedicated reconstruction algorithm be faster than the generic one. For this class, I did implement a dedicated reconstruction scheme before the general one. The coding process was very complex and the result requires numerous functions and is difficult to analyse. I kept it in the code base, but I did not measure its efficiency compared to the general method. This is because I found the simplicity of the general method preferable for maintenance reasons. 

The dedicated reconstruction, called \code{reconstruct\_old}, calls \code{evaluate} to obtain a minimal set of satisfying shares, if one exists, and then delegates the reconstruction work to \code{reduceLowestShares}.
This function is, and uses a number of, an auxiliary static function. This is the list, along with an explanation of how they work:

\begin{description}
\item[\code{reduceLowestShares()}] This is a complex function. It relies on the shares that it receives coming in the same order in which they were produced. The reconstruction follows a bottom-up approach. The important point is that shares are list in the order of a depth first traversal: if a share's identifier contains the previous share's identifier as a prefix, then it is a descendant of the previous one; if it has the same prefix as the previous one then the shares are siblings; otherwise, the new share is in a different branch of the tree and the shares before it can be used to reconstruct a secret until a point that shares a parent with the current share.

This function, like \code{findCoefficients}, uses a map to organize the nodes in the tree. Each key in the map is a node identifier, and is associated to a set of \cST. 

The function examines one share at a time and keeps track of a \code{currentSetPrefix} variable. This is the identifier of the parent node of the last share. The parent node of the current share is kept in \code{setPrefix}. The relation between these prefixes governs what is done in this method:

\begin{itemize}
\item If these prefixes are equal, then the current and the last share have the same parent and there should be already an entry in the map for that node, so the current share is added to it, by calling \code{updateSet}.
\item If \code{currentSetPrefix} is a prefix of \code{setPrefix}, then the new share is a descendant of the last one, and a new entry has to be created in the map, by calling \code{addNewSet}.
\item Otherwise, the new share starts a new branch at a higher level in the tree. The map then contains all the shares that could happen to the left of the current share and so these can be used for reconstruction until a node that has the same parent of the current share. This reconstruction is done by \code{reduceMapShares()}, which also receives the two prefixes so it can learn when to stop. It then does some checks to detect if the final node has been reached. That is the case if the prefix is \code{0} (the root) and there are no more shares in the map waiting to be processed. It also sets the variable \code{finished} to \code{true}.
\end{itemize}

It is possible to leave the loop traversing all shares without setting \code{finished}. This is the case when the last share's parent is a descendant or a sibling of the previous share. When this is the case, the function runs the final reconstruction by calling \code{reduceMapShares()}.



\item[\code{updateSet()}] this function retrieves a set of \cST from a map and adds a new element to it. Because the map structure returns values by copy and not by reference, it is necessary to reassign the value for that key, hence the instruction \code{  setShares[prefix] = set; }.

\item[\code{addNewSet()}] this function creates a new set of \cST with only one element and adds it to the map as the value for the key received as argument.

\item[\code{putShareInSet()}] this function tries to retrieve a vector of \cST from a map for a given key. If the key exists, it adds a new share to it, otherwise it calls \code{addNewSet} to create a new set for that share.

\item[\code{getSetPrefix()}] this function returns the set prefix of a share from its identifier, that is, the identifier of its parent node.

\item[\code{detailedReconstruction()}] this function creates a temporary share for an inner node of the policy tree, auxiliating to propagate the shares from the ground up until the secret at root level can be computed from its direct children. 

It reconstructs the secret for a set of shares corresponding to an simple Shamir node in the tree. It obtains the Lagrange coefficients and computes the linear combination of shares and coefficients. Since this is no longer a leaf node but yet must have a \cST the function computes a suitable identifier for this node. This is done by invoking \code{findIDForNode} and associating a \code{:=?} termination to signal the node as a pseudo-leaf.

\item[\code{solveSet()}] this function triggers the reconstruction of a secret for a particular node. It obtains from the map the set of shares for a given node identifier (\code{prefix}) and removes it from the map. It calls \code{detailedReconstruction} to obtain a new share that replaces that set and returns this share.

\item[\code{reduceMapShares()}] this function is used to reconstruct intermediate shares in the policy tree in a bottom-up approach, until it reaches a node that corresponds to a certain target identifier passed as argument.

Conceptually, it takes a few sibling shares hanging from an inner node, and replaces this by a pseudo-leaf node whose value is the secret reconstructed from those shares. This process goes on until the identifier of the new node's parent is equal to a target prefix.

It calls \code{solveSet} to reconstruct the lowest available set in the tree, as given by \code{prefix}, and receiving a new \cST. This is added to the map of shares, and \code{prefix} is updated to its own parent.

When the loop terminates, the function returns the last computed share.
\end{description}



\section{Comparison of Canonical BL and Shamir Tree schemes}

One of the fundamental thesis of [Crampton, Pinto 14] is that canonical BL can be implemented with much less effort than Shamir Tree, reducing the possibility of bugs and therefore possible security problems of the resulting scheme. I could witness this first hand during this project. Indeed, the canonical BL implementation was much simplified by the fact that its policy can be represented by a simple list of minimal sets. This makes the evaluation of the policy and the secret reconstruction particularly easy: one can linearly traverse the minimal sets and until one is found that is satisfied by the attributes present in the reconstructing set of shares. This can be done, at its most basic, by testing for each element of the target minimal set whether it is contained in the reconstructing set. This, again, can be done by a linear search. Other techniques can be used when the size or the number of sets is too large, for example using hash tables indexed by individual set elements, and containing as values other hash tables classified in the same form. This would have large space requirements but the structure would be easy to populate with sets (with a polynomial time overhead) and would provide a very fast search of a satisfied minimal set, if any.

In contrast, the policy of a Shamir Tree requires a tree like representation, or some complex way to linearize such a structure. The evaluation is recursive in nature, although it can be made iterative. However, this requires extra code for adequate structures, much more careful implementation and is a larger drain of mental resources for the coder. In this project, that was reflected in the need to create a dedicated data structure, the corresponding tests. Different ideas had to be tested for the tree representation, until an option was made for the use of smart pointers to adequately manipulate the tree. The effort for the programmer was much higher than needed for the implementation of the canonical BL version: I estimate to have spent at least twice or thrice as much time to implement the \cShTree and \cShTreeAP classes as for the corresponding canonical BL versions, plus the time needed for the tree imlementation. The main effort was in the procedures to reconstruct the tree with a bottom-up approach and in the procedure to compute the reconstruction coefficients.
The reason for this approach is an optimization suggested in GPSW06 for decryption in the ABE scheme. The na\"ive way to decryt is to first pair all key fragments with corresponding ciphertext fragments, and then work up the tree as these are combined. However, typically not all leaves are necessary in a successful reconstruction. By identifying a minimal set of leaves first, we avoid doing unnecessary pairing operations. 

A top-down approach would also be possible. The simplest case would traverse the tree in a depth-first manner, but unsatisfied leaves could only be identified at the lowest level. An alternative would be to use the information about the position of the necessary shares to curtail the traversal and only follow down paths that will be needed later. This however requires extra logic to keep track of the current position in the tree traversal, to search for the relevant information in the share public information and given the recursive nature of this descent would probably require a stack to reconstruct the value of a node from values held by descendants and that have not been computed yet. Altogether, it would be probably more complex than the current solution.

Besides the bottom-up reconstruction, I also implemented another variant, that could actually be used for the generic \cSS: compute the reconstruction coefficients with \code{findCoefficients} and then linearly combine these with the shares. Although the code for computing the coefficients could be similar to the initial reconstruction method, the implemented version uses two hash tables and a possibly longer running time to obtain a simpler, easier to understand (and maintain) code. After the difficulty of implementing the first reconstruction, I found this method preferable. However, I have not done tests to measure the relative efficiency of both methods. 

\subsection{Implementation via Monotone Span Programs}

It can be argued that the spirit of the schemes in GPSW06 was to use Monotone Span Programs (MSP) to directly reconstruct the secret, instead of evaluating the tree. This is debatable, as the authors proposed the aforementioned optimization in their conference version where there is no mention of MSPs. Furthermore, their decryption scheme does not separate the reconstruction of the secret from the computation of the plaintext. Nevertheless, I try to decouple the two below by giving a secret reconstruction from their decryption algorithm. This requires a representation of the LSSS by a  Monotone Span Program, that is, a matrix $M$ and a naming function $\rho$. $M$ contains a line for each share given by the scheme. 

Given these, the reconstruction implicitly outlined in GPSW06 would take the following steps:
\begin{enumerate}
\item \label{it_minset} identifying a minimal set of necessary shares $\alpha$: $\gamma$;
\item \label{it_submatrix}identifying a sub-matrix of $M$, $M_\gamma$ with only the lines corresponding to that set of shares;
\item \label{it_coefs}computing the coefficients for each share in $\gamma$;
\item for each line of the matrix $M_\gamma$:
\begin{enumerate}
	\item \label{it_matprod} compute the product between every line  and a fixed vector of random data; 
	\item \label{it_coefprod}multiply the result by the corresponding coefficient;
\end{enumerate}
\item \label{it_add} add the results for all the lines
\item \label{it_dotprod} add the components of the resulting line
\end{enumerate}

This is in fact very similar to the present implementation: steps \ref{it_minset}, \ref{it_coefs}, \ref{it_coefprod}, \ref{it_add} are included part of the generic reconstruction process given above. Steps \ref{it_submatrix}, \ref{it_matprod} are the implicit computation of the share, done in the key generation phase; \ref{it_dotprod} finally corresponds to adding all the different shares together.

It is known that any generic distribution algorithm uses a fixed set of random data and can create each share by computing independent linear combinations of these data. The matrix $M$ can then be seen as the collection of random coefficients of these linear combinations. The implementation in this project computes these shares in the key generation phase by invoking the secret sharing distribution algorithm, possibly using a more efficient algorithm than directly computing the linear combinations. But the reconstruction ends up doing exactly the same operations of this GPSW06 proposal, unless the secret sharing scheme's reconstruction allows for a simpler logic, \emph{as is precisely the case of the canonical BL}, where no coefficients have to be computed, or a possibly cheaper method, which might be the case of the first reconstruction algorithm for Shamir Tree. 
In any case, this implememtation allows the possibility of relying on a black box implementation of a secret sharing scheme, and possibly a more efficient implementation of the distribution function.

It should be noted that for decryption in the KPABE scheme, we actually need to obtain the reconstruction coefficients $\alpha$ themselves. But this is precisely the point the authors of GPSW06 do not address: such calculation is dependent on the concrete secret sharing scheme. And this gives a clear comparison between canonical BL and Shamir Tree: for the first, the coefficients of all shares in a minimal necessary set are all $1$; for the latter, these require some tree-like evaluation. The difference in complexity and logic effort is evident.


\section{Testing Environment}

All the classes and main functions used in this project are tested, albeit in an indirect way in the case of abstract classes. The testing code is consistent all across the project. Each testing file first begins with an \code{\#includes} call to the relevant header file, where the tested code is declared. Then, a \code{main()} function does some preparation, if necessary, and calls the function \code{runTests}. For \ftkpabe, \ftblcanon and \ftshtree, this preparation includes setting up the MIRACL library and some randomization. Furthermore, \fckpabe runs the function \code{runTests()} twice, each taking an instance of \cKPABE with a different secret sharing scheme but with the exact same policy. Before each call to \code{runTests}, it prepares the \cKPABE instance by running its \code{setup()} function.

The function \code{runTests} might take more or less arguments, depending on what is needed by the individual tests, but it has some invariants: it always initializes an variable \code{errors} to $0$ that counts how many tests have failed so far, and it always returns this value. Inside, \code{runTests} usually calls other functions that run specific tests and return their own count of errors. These functions, do all the heavy lifting, preparing and executing each test. One test corresponds to one call to the utility function \code{test\_diagnosis}, which takes three parameters: the message to output to the terminal signalling that a test was called; the condition that the test must satisfy, and the current error count. This function increments the counter if the condition fails, and outputs a message to the terminal saying that the test has either passed (in green) or failed (in red). 

Finally, \code{runTests} always calls the utility function \code{print\_test\_result} that shows how many tests have failed in the total package. If all tests pass, with colour coding for quick perception of the result.

\paragraph{Technical details}
This project was implemented and compiled in C++, over Linux Kubuntu 12.04.  The compiler used was gcc version 4.6.3 (Ubuntu/Linaro 4.6.3-1ubuntu5), for a target architecture \verb|x86_64-linux-gnu|. The target computer had a quadcore Intel CPU, i5-3210M running at 2.50 GHz, and 4GB of RAM memory.

The benchmark tests were ran overnight, as the sole user process running on the machine.


\section{Performance Measurement}

One of the objectives of this project is to test relative performance of canonical BL and Shamir Tree for policies represented in canonical form. The gathering of data for this analysis is implemented in file \code{benchmark.cpp}. It is used to obtain time measurements for each of the basic operations of a KPABE scheme. This file is used to create four different variants, for two different types of secret sharing scheme and two different ways of generating ciphertext and key fragments in the KPABE scheme. 

The specific configuration chosen is dictated by the constants that are defined, and these are declared in pairs of auxiliary header files. The choice of how to form attributes is defined in file \code{atts.h}, which is indirectly included via file \code{kpabe.h}. This is only relevant when the corresponding object file \code{kpabe1.o} or \code{kpabe2.o} is linked to the compilation of \code{benchmark.cpp}. Which one is used is dictated by the appropriate \fmake command. The alternative files \code{kpabe1.o} and \code{kpabe2.o} are also created by appropriate \fmake commands, that ensure that \code{atts.h} is created in the proper way: for \code{kpabe1.o}, \code{atts.h} is a copy of \verb|atts.h_1| and analogously for \code{kpabe2.o}. 
The file \code{atts.h} can be deleted at will as long as the files \verb|atts.h_1| and \verb|atts.h_2| are not, since the \fmake command will always reconstruct the former.

The other regarding the type of secret sharing scheme to use is done by a similar scheme, via the header file \verb|benchmark_defs.h|. This is a copy of either \verb|benchmark_defs_bl.h| or \verb|benchmark_defs_sh.h| and the appropriate \fmake command will choose which of these files supplies the contents of \verb|benchmark_defs.h|. This file will then include the appropriate header file for the chosen secret sharing scheme and define the constants \verb|SS_TYPE| and \verb|SS_ACC_POL_TYPE| that simply replace the name of the classes for a \cSS and an \cAP respectively.

The first task of \code{main()} is to setup the MIRACL library and seed the randomization process. Then, it invokes \code{parseInput} to check which switches the command was invoked with. These define what operations should be measured. Although there are four basic operations, there are 13 switches, corresponding to 12 specific measures plus the switch \code{all}, which runs all of them.

The reason of this discrepancy is that there are 5 variants each of key generation and decryption. Setup and Encryption are straightforward operations, but the others require a policy and \emph{a priori} different kinds of policies might have an influence in the time necessary for the operation. Since it is impossible to test all kinds of policies, I have chosen 5 kinds that are defined by two parameters: the maximum number of shares in the policy (referred in the code as leaves, since these are the leaves of the policy tree. Note that a canonical BL can also be represented as a tree, although this is not the most efficient representation) and another parameter $k$ of varying purpose. All the policies generated are naturally represented in canonical form. 

Types of policies:
\begin{description}
\item[Uniform] in this policy, the leaves are grouped in minimal sets of equal size. This size is determined by the parameter $k$. 
\item[Linear] in this policy, the size of minimal sets follows a linear progression, where the ratio is set by the parameter $k$. The first set has only one element.
\item[Exponential] in this policy, the size of minimal sets follows a geometric progression, where the ratio is formally set by paramenter $k$, but which is fixed at $k=2$. 
\item[Inverted Linear] this policy is just like the linear case, except that the size the sets decreases to $1$ instead of increasing from $1$.
\item[Inverted Exponential] this policy is just like the exponential case, except that the size the sets decreases to $1$ instead of increasing from $1$.
\end{description}

Each measurement tests 5 main variants of one parameter. In some cases, each variant will have other sub-variants. These are explained below.

The general idea to measure the time of an operation is to obtain from the system the exact time before the operation starts and the time right after it ends. The basic function to return the current time, \code{time()}, returns a value measured in seconds, which is too large to estimate times of much finer resolution. 

An alternative function, \code{clock()}, measures time in clock ticks. A macro \verb|CLOCKS_PER_SEC| represents the number of clock ticks per second for the system in question. I did some preliminary tests with this function and I did not obtain reliable results, usually receiving half the actual time spent (measured by an external clock). Because of this, I decided to use \code{time()}, which proved to be consistent with external clocks. My first tests were done with basic operations, implemented in file \code{basic-benchmark.cpp}. Some of these operations took in the order of microseconds, others in the order of milliseconds, but in any case always less than seconds. 

To achieve a proper measurement, I took the strategy or repeating a given operation a sufficiently large number of times (usually at least a few hundred, and when possible in the low thousands). The total time for this batch of executions is then divided by the number of iterations, to obtain the average operation time. This also has the advantage of smoothing out differences in each iteration by returning an average.

\subsection{\code{measureSetup()}}

This function measures the amount of time needed for a setup operation. This operation generates some fixed values, like the key randomness and its public encoding as an element of $GT$, and the public and private values for each attribute. The number of attributes is the only variable quantity in this process. This function runs 5 scenarios, with the following number of attributes in the universe: 5, 10, 100, 500, 1000.
The number of repetitions for each scenario is stored in variable \code{varRepeats}.

\subsection{\code{measureEncrp()}}

This function measures the amount of time needed for an encryption operation. This requires a fixed operation, a multiplication in the pairing target group. Plus, it requires the generation of one fragment for each attribute associated to the encryption. 

Again, 5 scenarios are tested, with the number of ciphertext attributes varying through 5, 10, 100, 500, 1000. This measures two ways to do encryption: one is the original encryption method, detailed in the literature since Sahai Waters 05, where the plaintext and ciphertext are elements of the group $GT$. 

The other method is taken from the sample implementation of Fuzzy-IBE in the MIRACL library, where a message is a string of limited size (as dictated by the size of a \cBig instance in MIRACL) which is interpreted as a \cBig. I saw no rationale in the documentation to MIRACL as to why this change should have been made, and so I decided to obtain performance data for each.

Accordingly, each iteration executes two encryptions where only the plaintext varies (and the internal randomness of the encryption, of course). The traditional plaintext is stored in a variable \code{M} of type \code{GT}, and is initialized to be a random value. Since MIRACL does not offer a method to obtain a random element of \code{GT}, this function first computes a fixed element of \code{GT} by pairing the parameters \code{P} and \code{Q}, of types \code{G1} and \code{G2} respectively and then raises this to a random \cBig integer. 

The MIRACL-like plaintext is instead always the string \code{"hello world to be encrypted"}. This is transformed into a \cBig and stored in variable \code{sM}. To do this transformation, we execute first \code{mip->IOBASE=256}. This parameter tells MIRACL how to convert a \cBig into a string, for example for printing its contents to the screen, or how to initialize a \cBig from a string, by defining the numerical base in which all \cBig are represented. Usually, this value is 16, so that a \cBig is written in hexadecimal form. By using base 256, a \cBig is instead represented as a series of bytes, which allows MIRACL to instantiate an integer from a page of text, simply taking the ASCII code of each letter as the value of the corresponding byte. 
After this conversion, the code simply sets the base back to the usual 16.

\subsection{\code{measureKeyFunc(...)}}

%As explained above, the code measures different policies for key generation. Aside from this difference, the code should be exactly the same in every case. This function implements the general structure of the code for every policy, using its arguments to provide it with the different policies that need to be tested. Before explaining the individual cases, I explain the general structure.

The code tests 5 scenarios, defined by the maximum number of shares in the policy. These are passed in as the array \code{leavesInPolicy}, and an indication of its size in \code{nLeavesVars}. The number of repetitions for each of these cases is passed in array \code{varRepeats}. Each type of policy allows several different variants: for example, we might have a uniform policy with sets of size $1$, or $100$. The general code does not know how many policies it should produce, now with what parameters. Instead, it uses data and functions passed in from the outside to do this:
\begin{description}
\item[\code{makePolicy}] is a  pointer to a function that generates a policy. Whatever this function is, it receives two parameters represented by variables \code{nLeaves} (which is one of the scenarios described in \code{leavesInPolicy}) and a parameter $k$. It also returns a policy as a \code{std::string}, and modifies the value of two outbound variables passed to it: \code{realNLeaves} and \code{realNSets}, that indicate, respectively, how many leaves and minimal sets are in the generated policy.
\item[\code{start\_k}] is an argument received by \code{measureKeyFunc(...)} that gives the starting value of the parameter $k$.
\item[\code{stop(...)}] is a pointer to a function that receives parameters \code{nLeaves} and \code{k} and decides either to produce a new policy with these parameters or stop the loop and advance for the next scenario.
\item[\code{next(...)}] is a pointer to a function that receives the parameter \code{k} and returns the next value of \code{k}, to use in the next iteration of this loop.
\end{description}

While the function \code{stop(...)} allows the \code{for} loop to be executed, \code{makePolicy} is invoked to return a new policy, which is then used to create a new access policy, corresponding secret sharing scheme and finally KPABE scheme. After initializing the latter, the code runs a batch of key generations.

I now give details on how to produce each policy.

\paragraph{Uniform policy}
The scenarios for this type of policy have the following numbers of leaves: 8, 32, 128, 512, 1024. 
The initial value of \code{k} is $1$, and at each step this is multiplied by $8$. Therefore, the number of leaves in a set always evenly divides the maximum number of leaves, and so the real number of leaves is always the maximum.

The function \code{makeKeyUnifPolicy} is used to produce a uniform policy. It does so by creating minimal sets of $k$ consecutive integers starting at multiples of $k$. No element appears in more than one minimal set. The creation of minimal sets is handled by the function \code{makeMinimalSet}, which is also called from the functions for other policies.

\paragraph{\code{makeMinimalSet}}
This is the function responsible for creating minimal sets for policies, based on two parameters: \code{first} is the value of the first element to include in the set; \code{codedLength} indicates how many elements should be in the set.

The parameter \code{codedLength} can be positive or negative. When it is positive, it simply indicates how many elements to include in the set, and that the elements should be consecutive and increasing from the first element. When it is negative, its absolute value is the number of elements to include in the set. These elements should be consecutive, but \emph{decreasing} from the first element. The order in which the minimal sets are concatenated is also reversed, with larger sets appearing before the previous ones. This has the consequence that when not all elements are used in a policy, those missing are the smallest ones.

\paragraph{Linear policies}
These policies have the same maximum numbers of leaves as the uniform policies. However, the structure of the minimal sets is different. Each minimal set has \code{k} more elements than the previous one. The value of \code{k} starts in $2$ and increases by $3$ while it stays under $10$ (as defined by functions \code{nextLinPol()} and \code{stopLinPol}). In practice, it takes values $2$, $5$ and $8$. 

The policy only uses as many minimal sets as it is possible to have without exceeding the maximum of the current scenario. This results in irregular values of leaves and minimal sets in the successive policies, achieving the possible maximum only for 1024 leaves when the increment parameter is $2$. The first minimal set always has the single element $0$. Consecutive minimal sets will have consecutive values of elements without ever repeating one of them. An example policy has the sets $(0), (1,2,3), (4,5,6,7,8), \ldots$.

The inverse linear policies follow the same general logic, except that now the smallest set is at the end of policy and not at the start. Its single element is the largest element in the policy, and no longer $0$. Each set has $k$ more elements than the following set, and so the first set is the largest but has the smallest elements. One example policy has the following sets (\code{nLeaves = 32, k = 2}): 
$(15,14,13,12,11,10,9,8,7),(22,21,20,19,18,17,16),(27,26,25,24,23),(30,29,28),(31)$.

\paragraph{Exponential policies}
These policies also have the same number of leaves as the other kinds. However, only one value of the parameter \code{k} is considered: $2$. In these policies, the first set always has the single element $0$. Consecutive sets have double as many elements as the previous. Since the number of leaves in each set is a geometric progression starting in $1$, and the maximum number of leaves is a pure power of $2$, the resulting number of leaves is exactly the maximum minus $1$.

Like the inverse linear policies, the inverse exponential have the same structure of their non-inverse relative, but their smallest minimal set is at the end and has the highest element. Each minimal set has double the amount and smaller elements than the next set. An example policy is 
$(16,15,14,13,12,11,10,9,8,7,6,5,4,3,2,1),(24,23,22,21,20,19,18,17),(28,27,26,25),(30,29),(31)$.

\subsection{\code{measureDecFunc(...)}}
This function measures the decryption policy. As is the case for Key Generation, decryption tests different kinds of policies. The exact same functions used for Key Generation are also used here. Therefore, I will only describe the general behaviour.

This function starts by preparing plain texts to encrypt. As all the other operations, it has 5 basic scenarios for the number of leaves in the policy. These are passed in array \code{leavesInPolicy}, whose size is passed in \code{lvsInPol}. However, since encryption can take a variable number of attributes, each of these basic scenarios has 5 further sub-scenarios, where the number of attributes in the cipher text varies through 10, 50, 150, 600, 1024.

Before any of the scenarios begin, this function prepares two plaintexts, exactly as described in \code{measureEncrp}. It measures the decryption of both plaintexts, testing an eventual difference between the two encryption/decryption techniques.

Then it starts an outer loop that represents the basic scenarios. For each of them, it prepares policy variants according to the functions passed as arguments. After getting an adequate policy, it creates a set of elements that are authorized to decrypt the ciphertext, that is, that contains some minimal set. The way to create such a set depends on the particular policy used and the main work is delegated to the function pointed to by \code{findMiddleSetSizeAndFirstElement}, which returns the first element of the set in the outbound variable \code{begin} and the intended size of this set, as the return value.

Finally, given that for a given scenario the average execution varied significantly according to the size of this decrypting set, stored in variable \code{minSetSize}, the true number of repetitions is scaled for each policy, by dividing \code{varRepeats[i]} by \code{minSetSize}.

At this point, a new loop is started to run the timed batch of operations for each sub-scenarios.

\paragraph{Creating the decrypting set of attributes}

Each type of policy creates a different kind of decrypting set, using a different function. These functions receive three parameters: the maximum number of leaves, the real minimal sets in the policy, and the parameter \code{k} used to produce the policy. The functions' return value is the size of the decrypting set to be produced, while an outbound variable records the first element of that set. 

\begin{description}
\item[\code{middleSetUnifPol}] This function is used for uniform policies. It returns a size equal to that of the minimal sets in the policy. For the beginning element, it first finds the index of the set occupying the middle position among all sets, if their number is odd (first set's index is $0$) or of the first set after the middle point, if their number is even. Then, it returns the element occupying the first position of this set.
Thus, these parameters allows the exact reconstruction of some minimal set of the policy.

\item[\code{middleSetLinPol}] This function is used for non-inverted linear policies. As for uniform policies, it computes the index of the middle set or the one after the middle point. It computes the exact size of this minimal set, with help of the arithmetic progression ratio used to build the policy in the first place. Then, it computes the first element in that set. 

By construction, the first element of the set with index $i$ is the total of elements in all the previous sets. And since these increase linearly, the size of the $i^{th}$ set ($ i \geq 0$) is $1 + i \cdot k$, for some increment $k$. The first element of the $i^{th}$ set is then: $\sum_{j=0}^{i-1} (1 + jk) = i + i(i-1) k /2$.

Thus, this function also reconstructs exactly a minimal set in the policy.

\item[\code{middleSetLinPolInv}] This function is used for inverted linear policies. It starts by computing the index of the middle set.  The logic to compute the size and the first element, however, is more complicated. Given that the linear progression is in reverse, the smallest set, with size $1$, has index \code{nSets - 1}. Therefore, the size of the $i^{th}$ set is now $1 + (nSets - 1 - i) \cdot k$.

In each set, the first element is the largest element that has not been used in any smaller set. The last element is the first element of the previous set plus one. The elements on the set don't have to be ordered, so this function computes the value of the smallest and therefore last element in the set.

The largest element is the maximum number of leaves minus $1$, and so is always known. Then, the smallest element in set $i$ is this largest number minus the sum of the sizes of all sets with index equal or larger than $i$ up to \code{nSets - 1} minus $1$. Since the size of the last set is precisely $1$, this sum reduces to $$\sum_{j=i}^{nSets - 2} (1 + (nSets - 1 - j) \cdot k) = (nSets - 1 - i) \left(1 + k \cdot \left(nSets - 1\right) - k \cdot \left( nSets - 2 + i\right)/2 \right).$$

This produces exactly one of the minimal sets in the policy.

\item[\code{middleSetExpPol}] This function is used for exponential policies. Like its counterparts, it starts by computing the index of the middle set. The size of the set is simply computed as the power of $k$ with that index as exponent. The first element of set $i$ is the count of all elements that came before, which corresponds to the sum of all powers of $k$ before $i$. This is $(k^i - 1)/(k-1)$.

This produces exactly one of the minimal sets in the policy.

\item[\code{middleSetExpPolInv}] This function is used for inverse exponential policies. The index of the middle set is computed in the same way. The size of the corresponding set obeys the same exponential function, only the exponent is instead the difference between the index of the target set and that of the first set.

The value of the smallest element in the $i^{th}$ set now is equal to the maximum number of leaves minus the sum of all powers of $k$ from exponent $0$ to \code{nSets-1 - i}. This sum is $(k^{nSets-i}-1)/(k-1)$.

This produces exactly one of the minimal sets in the policy.
\end{description}



\end{document}